\section{Introduction}
In this mid-term report, updates are provided regarding the Directed Random Weight Graphs project. In relation to what is reported in the project proposal, the work is proceeding in compliance with what is stated in the objectives section. 
The class related to the Directed Random Weight Graph has been defined within the Jupyter notebook. Some functions for the correct management and visualization of the graph have been defined within the class.
The concept of weighted in-degree and out-degree for a single vertex and for the entire graph has been formalized.

Through a sampling process, instances of the graph are generated to approximate the weighted degree of each vertex. The number of samples generated is equal to the value determined by the Cochran formula (reference).
Subsequently, the calculation of the maximum between a set of random variables has been formalized using an exact approach and an approximate approach through the Gumbel distribution (mettere reference e scrivere la motivazione per la quale è stato fatto). 

The experimental results obtained to date are displayed within the notebook for a comparison through the use of charts. 
An algorithm for calculating shortest paths has been designed and its implementation has begun (spiegare meglio che è una variante del Bellman-Ford o comunque dire che ci basiamo su un approccio Greedy) 

For the final report, it was planned to (spiegare cosa si vuole fare e cosa ci siamo prefissati di fare, quindi performance profile ecc) 