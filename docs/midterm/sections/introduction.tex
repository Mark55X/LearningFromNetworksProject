\begin{projsection}{Introduction}
In this mid-term report, updates are provided regarding the Directed Random Weight Graphs project. In relation to what is reported in the project proposal, the work is proceeding in compliance with what is stated in the objectives section. 

The class related to the Directed Random Weight Graph has been defined within the Jupyter notebook. Some functions for the correct management and visualization of the graph have been implemented.

The concept of weighted in-degree and out-degree for a single vertex and for the entire graph has been formalized.
A similar analysis has been conducted for the cycles.

%Through a sampling process, instances of the graph are generated to approximate the weighted degree of each vertex.
%The number of samples generated is equal to the value determined by the Cochran formula (reference).
To achieve these results, several probabilistic concepts have been explored, including the calculation of the maximum among a set of random variables.
The report will present two different approaches for the computation of the maximum: an exact one and an approximation based on the Gumbel distribution.

The obtained experimental results will be displayed in the notebook through the use of charts. 

A new algorithm for calculating shortest paths has been designed.

%and will be presented its implementation has begun (spiegare meglio che è una variante del floyd-warshall o comunque dire che ci basiamo su un approccio Greedy) 
%spiegare l'idea di base (analisi statica e dinamica) e dire che abbiamo implementato l'algoritmo come variante di fw e che ora faremo i test e analisi di performance.
Up to now, the more theoretical aspects have been analyzed; the future work will focus on the practical analysis, particularly on the performance of the newly proposed algorithm. Furthermore, according to the theoretical considerations addressed, the aspect of cycle search will be explored in depth, especially those with positive sum. 

Based on the performance achieved, the algorithm will also be evaluated on real data. 
%Up to now the more theoretical aspects have been analized, the future work will concentrare on the practical analysis, in particular on the performances of the new proposed algorithm.
%For the final report, it was planned to (spiegare cosa si vuole fare e cosa ci siamo prefissati di fare, quindi performance profile ecc) 
\end{projsection}

%to do: ricerca e analisi dei cicli
