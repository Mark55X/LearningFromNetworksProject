\section{Real Data Analysis}\label{experiments}
In this section, the results of the proposed project will be discussed on real data downloaded from the Binance platform.
\subsection{Data Retrieval}
A Python code was created to download all trades between different cryptocurrencies. It was decided to download transactions for a single month of the year 2024 to manage memory usage and reduce variance in the data.
The \texttt{networkx} library allows you to save the structure of a graph inside semi-structured files; in this case, an XML file called \texttt{01\textunderscore01\textunderscore 24.graphml} is used. 
After downloading all the data, with the \texttt{write\textunderscore graphml} function, a weighted graph is defined in which the vertices are the cryptocurrencies and the weight of the edges is a tuple composed by the mean and standard deviation of each exchange rate between each currency.
Once the file is saved, it is read inside the Jupyter Notebook through the \texttt{read\textunderscore graphml} function in order to adapt the graph to the DWRG class proposed by the project. Consequently the final graph is created to perform the analysis.

\begin{figure}[h!]
	\centering
	\includegraphics[width=0.5\textwidth]{../../code/notebooks/images/01_01_2024.png} % Replace with your image filename
	\caption{Binance Graph}
	\label{plot}
\end{figure}

\subsection{Experiments}
The experiments performed on the graph, were performed on a sample of the graph.
The first experiment that was performed, was done by looking for the shortest path with the greatest number of arcs. The result reports WIN as the starting currency and ARS as the target currency, with a path 8 arcs long. Once this pair of vertices is obtained, a comparison is performed between our greedy algorithm and an exact algorithm, in this case the Bellman-Ford (since there are negative arcs).
From the results obtained, it can be noted that the exact algorithm returns a minimum path of length 8 with a cost equal to 2.4654882457904144. Instead, our algorithm returns a path of length 5 and a cost equal to -2.408182003102148. 
Performing the same experiment on another pair of currencies, AAVE and TRY a similar result is found.
In particular, the exact algorithm returns a shortest path of length 6 and cost 7.915443742635086, while our algorithm finds a path of length 3 and cost 7.95438190729025.
It can be said that exploiting the local optimality, given by the estimate of the distances calculated by the greedy algorithm, leads to a better result than the exact algorithm.

In the second experiment, we analyze the shortest paths among all pairs of vertices (currencies) that are connected by an arc.
We focus on those with a length (rather than weight) greater than 1.
In other words, through this experiment it is possible to identify currency pairs for which a direct exchange is inconvenient compared to making multiple exchanges between them.
From the results obtained, 614 pairs were identified for which this situation is present. In particular, between the UMA currency and USDT, a direct exchange has an exchange rate equal to 5.0, instead, passing first through BTC, the final exchange rate is equal to 10.32. Consequently, a non-direct exchange between these two currencies could double their value.

From the different samples obtained, no cycles were identified within the graph and consequently no experiments were performed in this regard. To perform analyses of this type, a possible solution could be to acquire a greater number of data in order to obtain a graph as similar as possible to a complete graph. In this way, a bidirectional connection between each currency pair would be present, and consequently, the analysis of cycles would be feasible.
Future studies could be interesting to be able to extend this type of graph also to different probabilistic distributions, in order to analyze different real scenarios, in which the connections between the different nodes fluctuate over time.
