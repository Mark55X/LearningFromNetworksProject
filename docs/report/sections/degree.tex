\section{Weighted degree}
Let $G=(V,A)$ be a sample of the DRWG and $u$, $v$ $\in V$. Since each arc $(u,v)\in A$ whose weight is $ W(u, v) \sim \mathcal{N}(\mu_{(u, v)}, \sigma^2_{(u, v)}) $, we define:
\begin{itemize}
	\item The \textit{weighted out-degree} for the node $u$ is:
	\begin{align*}
		d_W^+(u) = \sum_{v \in \mathcal{N}(u)} W(u, v) \sim \mathcal{N}(\sum_{v \in \mathcal{N}(u)} \mu_{(u, v)}, \sigma^2_{(u, v)})
	\end{align*}
	\item The \textit{weighted in-degree} for the node $u$ is:
	\begin{align*}
		d_W^-(u) = \sum_{v \in \mathcal{N}(u)} W(v, u) \sim \mathcal{N}(\sum_{v \in \mathcal{N}(u)} \mu_{(u, v)}, \sigma^2_{(u, v)})
	\end{align*}
\end{itemize}

Where $\mathcal{N}(u)$ is the set of neighbours of node $u$.
Fixing a node $u$, the computing time for $d_W^+(u)$ and $d_W^-(u)$ is 

\subsection{Degree of the graph}
The \textit{degree of the graph} $G$ has been defined as:
\begin{align*}
	d^+_w(G) = max_{u\in V}(d_W^+(u))
	d^-_w(G) = max_{u\in V}(d_W^-(u))
\end{align*}

Essendo il massimo di variabili aleatorie, è necessario utilizzare i metodi definiti nella sezione 2.



