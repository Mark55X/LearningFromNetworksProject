\documentclass[12pt, letterpaper]{article}

%{Environments
\newenvironment{projsection}[1]{
	\noindent \textbf{#1:}
}{
	% \par
	\\[.3cm]
}
%}

%{Custom commands
\newcommand{\blank}{
    \newpage
    \null
    \newpage
}
\newcommand{\fnt}[1]{
    \fontfamily{#1}\selectfont
}
%}

\begin{document}

	\title{
	\vspace{-2cm}
    \textbf{Modeling and Analysis of Directed Random Weighted Graphs: Applications to Cryptocurrency Networks}\\[.4cm]
    
    {\large\emph{Learning From Networks - Project Report}}
    {\large\texttt{https://github.com/Mark55X/LearningFromNetworksProject.git}}
}

\author{
  Luigi Frigione - 2060685\\
  \texttt{\small{luigi.frigione@studenti.unipd.it}}
  \and
  Marco Stefani - 2139662\\
  \texttt{\small{marco.stefani.3@studenti.unipd.it}}
  \and
  Edoardo Parpaiola - 2139221\\
  \texttt{\small{edoardo.parpaiola@studenti.unipd.it}}
}

\date{\today}

\maketitle

%{if abstract
\begin{abstract}
This paper introduces a novel model for weighted directed random graphs, in which the graph structure (i.e., the vertices and arcs) is fixed, while the arc weights are treated as random variables.
Various key features of the network are analyzed within this framework, progressing from simple structures to more complex ones.

The study begins with an examination of the weighted degree of a node.
Next, the concept of shortest path is explored and a new greedy algorithm is proposed to compute it.
Finally, to analyze the graph structure and the proposed algorithm, some experiments are performed in a real-world scenario using data retrieved from Binance.
\end{abstract}

\vfill
{\fnt{qtm}
    \emph{Professor: Fabio Vandin}
}

\newpage

\tableofcontents
%}

%{if not abstract
% \tableofcontents

% \vfill
% {\fnt{qtm}
%     \emph{Professor: Fabio Vandin}
% }
%}


	\begin{projsection}{Title}
	Esempi di titoli per la nostra rete:
	\begin{itemize}
		\item in generale prò essere: PROBABILISTIC NETWORK
		\item oppure Distributed Uncertainty Network
		\item GPN: gaussian probabilistic network 
	\end{itemize}
	
	Come titoli della ricerca:
	
\end{projsection}
	\begin{projsection}{Motivation}
	To date, most of the graphs analyzed in the literature concern structures in which the weights of the arcs are numerical values fixed over time, or graphs in which the edges have a certain probability associated with their existence between each pair of vertices \cite{Exp_Random_Graph}. However, this type of graph cannot be adopted in any application scenario. Imagine a graph in which each node represents a currency, physical or digital, and the arcs represent the exchange rates between each pair of currencies. To perform a correct analysis of this type of graph, fixed values cannot be assigned to the arcs, since their weight can fluctuate over time following a certain probability distribution.
	
	Consequently, it is clear that there is a need to define a type of graph that does not coincide with traditional probabilistic graphical models, such as Bayesian Networks or Markov Chains \cite{Probabilistic_Graphical_Models}.
	Direct Random Weighted Graph can represent a new field of research, in which its structure is well defined and stable, with all the information contained in the variability of the relationships between the vertices.
	Following this direction of study allows the analysis of different fields characterized by time-varying relationships. Among the possible scenarios, one case could be the financial one, where this type of graph can explore currency arbitrage opportunities, identifying positive-sum cycles within the graph, generated by discrepancies in the quoted prices of the currencies, in order to capitalize on them \cite{CurrencyArbitrage}.
	Treating exchange rates as probability distributions allows us to simulate market fluctuations and study how they can influence the graph, propagating their effects to other currencies or markets.
	
	
\end{projsection}
	\begin{projsection}{Method}
	\begin{itemize}
		\item Definizione del grafo probabilistico
		\item Definizione di distanza tra due nodi
		\item Dimostrazione che è possibile applicare djkstra
		\item Metodo perfetto per il calcolo della distanza (misurare il tempo in base al numero di nodi)
		\item Metodo Sampling e diversi esperimenti per calcolare quanti sampling servono per ottenere il valore perfetto (verificare le tempistiche)
		\item Metodo approssimato per grandi reti: sampling dei rami per ridurre la complessità e i tempi
		\item Dimostrazione che l'aspettazione dell'approssimato è uguale al calore teorico
		\item Confronto generale
		\item Calcolo del diametro come esempio dell'importanza nel definire i metodi di calcolo della distanza
		\item Migliorie future
	\end{itemize}
\end{projsection}
	\begin{projsection}{Intended experiments}
A specific dataset is not required to construct the graph, as the primary focus is on testing theoretical aspects. However, in the financial sector, the trading platform Binance provides detailed transaction data for cryptocurrency trades \cite{Binance}.
In particular, there are CSV files containing daily transaction data, as well as monthly summaries, for various cryptocurrency trading pairs. Each CSV file provides detailed information on individual trades, such as timestamps, prices, and trade volumes. These files can be used to define possible probability distributions for crypto trades which will then be assigned to the arcs of the graph under study.
As previously mentioned, random data can also be generated to model well-known distributions, such as Gaussian (or normal) distributions, by specifying parameters like the mean and standard deviation. This approach enables the simplification of complex problems by focusing on statistical approximations rather than precise data.

The code for computing the graph features will be written in Python and will be presented in a Jupyter notebook. This notebook will include the generation of the graph as well as all the experiments, accompanied by relevant plots like box-plots, histograms. The results and findings will be presented in a mathematical format, along with the corresponding pseudocode, in the final paper. 
The decision was made to use the \texttt{networkx} library version 3.4.2 \cite{Networkx}, an open-source Python library whose first versions date back to 2014 and is still being actively updated, with over 700 contributors. This library was chosen because it allows us to easily create graphs (both through code or by importing them from files), with the ability to assign any object to each arc (in our case, the parameters of the probability distribution or an object representing the actual distribution). The library also allows for automatic plotting of the graph and includes methods with the implementation of well-known algorithms, such as the one for the shortest path.

\end{projsection}

	\section{Note Marco}
	\subsection{API exchange}
	\begin{itemize}
		\item BINANCE: forse il più attendibile essendo motore per comprare/vendere crypto. Richiede di aver versato dei soldi nel conto.
		https://developers.binance.com/docs/binance-spot-api-docs/rest-api#exchange-information
		\item https://github.com/fawazahmed0/exchange-api API gratuita, che fornisce dati solo una volta al giorno. No login. Il problema è che non dice la fonte...
		saranno dati anche veritieri ma ci serve la fonte.
		\item https://exchangerate.host/documentation è un API che aggiorna i suoi dati ogni 10 minuti secondo la documentazione. Richiede accesso con tokern gratuito massimo
		100 richieste, contiene le Historical Rates 
		\item Anche quesot è un dataset https://www.kaggle.com/datasets/dhruvildave/currency-exchange-rates interessante...con exchange di 130 valute. Da capire la validità. Ci sono qualche errori come si può notare da qua https://www.kaggle.com/discussions/general/234811 non so se siano stati risolti...
		Sarebbe da approfondire se colossi danno i dati sottoforma di dataset perche tanto ci serve lo storico. 
		\item https://www.kaggle.com/datasets/kaushiksuresh147/top-10-cryptocurrencies-historical-dataset Bel dataset ma come base solo i dollari
	\end{itemize}
	
	
	% \bibliographystyle{plain}
\bibliography{bibliography/refs}


https://data.binance.vision/?prefix=data/spot/

https://github.com/networkx/networkx

\nocite{*}

\end{document}