\begin{projsection}{Intended experiments}
For this project, a specific dataset is not required to construct the graph, as the primary focus is on testing theoretical aspects. However, in the financial sector, the trading platform Binance provides detailed transaction data for cryptocurrency trades \cite{LINK URL}
In particular, there are CSV files containing daily transaction data, as well as monthly summaries, for various cryptocurrency trading pairs. Each CSV file provides detailed information on individual trades, such as timestamps, prices, and trade volumes. These files can be used to define possible probability distributions for crypto trades which will then be assigned to the edges of the graph under study.
As previously mentioned, random data can also be generated to model well-known distributions, such as Gaussian (or normal) distributions, by specifying parameters like the mean and standard deviation. This approach enables the simplification of complex problems by focusing on statistical approximations rather than precise data. \\\\

The code for computing the graph features will be written in Python and will be presented in a Jupyter notebook. This notebook will include the generation of the graph as well as all the experiments, accompanied by relevant plots like box-plots, histograms. The results and findings will be presented in a mathematical format, along with the corresponding pseudocode, in the final paper. 
The decision was made to use the networkx library version 3.4.2 \cite{LINK_GITHUIB}, an open-source Python library whose first versions date back to 2014 and is still being actively updated, with over 700 contributors. This library was chosen because it allows us to easily create graphs (both through code or by importing them from files), with the ability to assign any object to each edge (in our case, the parameters of the probability distribution or an object representing the actual distribution). The library also allows for automatic plotting of the graph and includes methods with the implementation of well-known algorithms, such as the one for the shortest path.

\end{projsection}