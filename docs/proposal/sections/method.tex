\begin{projsection}{Method}
	\begin{itemize}
		\item Definizione del grafo probabilistico
		\item Definizione di distanza tra due nodi
		\item Dimostrazione che è possibile applicare djkstra
		\item Metodo perfetto per il calcolo della distanza (misurare il tempo in base al numero di nodi)
		\item Metodo Sampling e diversi esperimenti per calcolare quanti sampling servono per ottenere il valore perfetto (verificare le tempistiche)
		\item Metodo approssimato per grandi reti: sampling dei rami per ridurre la complessità e i tempi
		\item Dimostrazione che l'aspettazione dell'approssimato è uguale al calore teorico
		\item Confronto generale
		\item Calcolo del diametro come esempio dell'importanza nel definire i metodi di calcolo della distanza
		\item Migliorie future
	\end{itemize}
\end{projsection}