\begin{projsection}{Motivation}
	Fino ad ora, i grafi studiati negli ultimi anni riguardano grafi in cui i pesi dei rami sono dei valori numerici e fissi.
	Ci sono molte situazioni in cui però questa assunzione non può essere vera. Ad esempio, si immagini un grafo in cui
	i nodi rappresentano le diverse valute e i rami i tassi di cambio: come si può ben immaginare, per una corretta analisi della rete,
	non si possono definire dei valori fissi negli archi, che invece possono assumere dei diversi valori in base ad una distribuzione di probabilità.
	
	Il modello quindi che si vuole definire non combacia con modelli grafici probabilistici come sono le Bayesan Network o le Markov Chain ma rappresentano un ambito nuovo da poter esplorare per analizzare realà come la finanza, le reti valutarie, reti di arbitraggio in mercati finanziari (come l'individuazione di cicli a somma positiva per ottenere profitti). Modellare i tassi come distribuzioni probabilistiche permette di simulare come le fluttuazioni in una valuta o tasso possano trasmettersi attraverso la rete, influenzando altre valute o mercati. Questo tipo di modello può essere utilizzato per stress-testare l’intera rete e valutare la sua resilienza a shock finanziari.
	
	(Link alla chat di GPT per altre idee : https://chatgpt.com/share/672646c1-9444-800a-a1a9-9d7a84307239)
	
	
\end{projsection}