\begin{projsection}{Motivation}
	To date, most of the graphs analyzed in the literature concern structures in which the weights of the arcs are numerical values fixed over time, or graphs in which the edges have a certain probability associated with their existence between each pair of vertices \cite{Exp_Random_Graph}. However, this type of graph cannot be adopted in any application scenario. Imagine a graph in which each node represents a currency, physical or digital, and the arcs represent the exchange rates between each pair of currencies. To perform a correct analysis of this type of graph, fixed values cannot be assigned to the arcs, since their weight can fluctuate over time following a certain probability distribution.
	
	Consequently, it is clear that there is a need to define a type of graph that does not coincide with traditional probabilistic graphical models, such as Bayesian Networks or Markov Chains \cite{Probabilistic_Graphical_Models}.
	Direct Random Weighted Graph can represent a new field of research, in which its structure is well defined and stable, with all the information contained in the variability of the relationships between the vertices.
	Following this direction of study allows the analysis of different fields characterized by time-varying relationships. Among the possible scenarios, one case could be the financial one, where this type of graph can explore currency arbitrage opportunities, identifying positive-sum cycles within the graph, generated by discrepancies in the quoted prices of the currencies, in order to capitalize on them \cite{CurrencyArbitrage}.
	Treating exchange rates as probability distributions allows us to simulate market fluctuations and study how they can influence the graph, propagating their effects to other currencies or markets.
	
	
\end{projsection}